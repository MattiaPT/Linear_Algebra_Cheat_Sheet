\section{Eigenvalues and Eigenvectors}
\begin{mainbox}{General}
\setlength{\tabcolsep}{2pt}
\begin{tabular}{rl}
	Def. & We call the number $\lambda\in\mathbb{E}$ \textcolor{blue}{eigenvalue (EVal)} of the\\
	& linear transformation $F:\;V\rightarrow V$, if there exists an\\
	& \textcolor{blue}{eigenvector (EVec)} $v\in V$, $v\neq o$, such that: $F(v) = \lambda v$\\
	\rule{0pt}{3ex}
	Def. & We denote the \textcolor{blue}{eigenspace} $E_\lambda$, containing all\\
	& eigenvectors for $\lambda$:\quad $E_\lambda :\equiv \{v\in V | F(v) = \lambda v\}$\\
	& (the eigenspace is a subspace of V)\\
	\rule{0pt}{3ex}
	Def. & The set of all eigenvalues of F is called \textcolor{blue}{spectrum},\\
	& denoted $\sigma (F)$.\\
	\rule{0pt}{3ex}
	Def. & $\xi\in\mathbb{E}^n$ is an eigenvalue of of $\lambda$ $\Leftrightarrow$ $A\xi = \xi\lambda$\\
	\rule{0pt}{3ex}
	\textcolor{blue}{L 9.1} & A linear transformation F and it's matrix represent-\\
	& ation have the same eigenvalues and the eigenvalues\\
	& are related respecting the coordinate transformation.\\
	\rule{0pt}{3ex}
	\textcolor{blue}{L 9.2} & $\lambda$ is an eigenvalue, iff $ker\;(\text{A}-\lambda I)$ doesn't contain\\
	& just the zero vector (singular). $E_\lambda = ker\;(\text{A}-\lambda I)$\\
	\rule{0pt}{3ex}
	Def. & The \textcolor{blue}{geometric multiplicity} of $\lambda$ is $dim\;E_\lambda$.\\
\end{tabular}
\end{mainbox}
\section{Eigenvalues and Eigenvectors}
\begin{mainbox}{General}
\setlength{\tabcolsep}{2pt}
\begin{tabular}{rl}
	Def. & We call the number $\lambda\in\mathbb{E}$ \textcolor{blue}{eigenvalue (EVal)} of the\\
	& linear transformation $F:\;V\rightarrow V$, if there exists an\\
	& \textcolor{blue}{eigenvector (EVec)} $v\in V$, $v\neq o$, such that: $F(v) = \lambda v$\\
	\rule{0pt}{3ex}
	Def. & We denote the \textcolor{blue}{eigenspace} $E_\lambda$, containing all\\
	& eigenvectors for $\lambda$:\quad $E_\lambda :\equiv \{v\in V | F(v) = \lambda v\}$\\
	& (the eigenspace is a subspace of V)\\
	\rule{0pt}{3ex}
	Def. & The set of all eigenvalues of F is called \textcolor{blue}{spectrum},\\
	& denoted $\sigma (F)$.\\
	\rule{0pt}{3ex}
	Def. & $\xi\in\mathbb{E}^n$ is an eigenvalue of of $\lambda$ $\Leftrightarrow$ $A\xi = \xi\lambda$\\
	\rule{0pt}{3ex}
	\textcolor{blue}{L 9.1} & A linear transformation F and it's matrix represent-\\
	& ation have the same eigenvalues and the eigenvalues\\
	& are related respecting the coordinate transformation.\\
	\rule{0pt}{3ex}
	\textcolor{blue}{L 9.2} & $\lambda$ is an eigenvalue, iff $ker\;(\text{A}-\lambda I)$ doesn't contain\\
	& just the zero vector (singular). $E_\lambda = ker\;(\text{A}-\lambda I)$\\
	\rule{0pt}{3ex}
	Def. & The \textcolor{blue}{geometric multiplicity} of $\lambda$ is $dim\;E_\lambda$.\\
	\rule{0pt}{3ex}
	Def. & The \textcolor{blue}{characteristic polynomial} of A$\in\mathbb{E}^{nxn}$\\
	& is defined as $x_A(\lambda) :\equiv \text{det}(\text{A}-\lambda I)$. We call $x_A(\lambda) = 0$\\
	& the \textcolor{blue}{characteristic equation}.\\
	\rule{0pt}{3ex}
	\textcolor{blue}{L 9.4} & $x_A(\lambda) = (-\lambda)^n + Tr(\text{A})(-\lambda)^{n-1} + \ldots + \text{det(A)}$\\
	\rule{0pt}{3ex}
	\textcolor{blue}{L 9.5} & $\lambda\in\mathbb{E}$ is an eigenvalue of $A\in\mathbb{E}^{nxn}$\\
	& $\Leftrightarrow$ $\lambda$ is a root of $x_A$\\
	& $\Leftrightarrow$ $\lambda$ is a solution to the characteristic equation\\
	\rule{0pt}{3ex}
	Def. & The \textcolor{blue}{algebraic multiplicity} of an eigenvalue $\lambda$ is the\\
	& multiplicity of $\lambda$ as a root for the characteristic\\
	& equation.\\
	\rule{0pt}{3ex}
	\textcolor{blue}{L 9.6} & A singular $\Leftrightarrow$ $0\in\sigma(\text{A})$\\
	\rule{0pt}{3ex}
	\textcolor{blue}{T 9.7} & Similar matrices have the same characteristic equa-\\
	& tion, determinant, trace and the same eigenvalues.\\
	\rule{0pt}{3ex}
	Def. & A basis made of eigenvectors is a \textcolor{blue}{eigen basis} of F:\\
	& $x = \displaystyle\sum_{k=1}^n \xi_kv_k\quad\mapsto\quad F(x) = \displaystyle\sum_{k=1}^n \lambda_k\xi_kv_k$\\
	\rule{0pt}{3ex}
	\textcolor{blue}{T 9.9} & There is a similar diagonal matrix $\Lambda$ to $\text{A}\in\mathbb{E}^{nxn}$\\
	& $\Leftrightarrow$ A has an eigen basis $\qquad\textcolor{red}{\text{A}\text{V} = \text{V}\Lambda}$\\
\end{tabular}
\end{mainbox}

\begin{mainbox}{Spectral/Eigenvalue Decomposition}
\setlength{\tabcolsep}{2pt}
\begin{tabular}{rl}
	Def. & We call a matrix A to which a spectral decomposition\\
	& A$=\text{V}\Lambda\text{V}^{-1}$ exists \textcolor{blue}{diagonalizable}.\\
\end{tabular}
\end{mainbox}
\begin{mainbox}{}
\setlength{\tabcolsep}{2pt}
\begin{tabular}{rrl}
	\textcolor{blue}{T 9.13} & \multicolumn{2}{l}{geom. mult. $\leq$ alg. mult.}\\
	\rule{0pt}{3ex}
	\textcolor{blue}{T 9.14} & \multicolumn{2}{l}{A matrix is diagonalizable, if for every}\\
	& \multicolumn{2}{l}{eigenvalue holds: geom. mult. $=$ alg. mult.}\\
	\rule{0pt}{3ex}
	\textcolor{blue}{T 9.15} & \multicolumn{2}{l}{\textcolor{blue}{Spectral Theorem}: Let A$\in\mathbb{C}^{nxn}$ be hermitian:}\\
	& i) & all eigenvalues are real\\
	& ii) & the eigenvectors are pairwise orthogonal\\
	& iii) & there is an orthonormal basis of $\mathbb{C}^n$\\
	& & consisting of eigenvectors of A\\
	& iv) & for this unitary matrix U holds:\\
	& & $\text{U}^\text{H}\text{AU} = \Lambda :\equiv \text{diag}(\lambda_1, ..., \lambda_n)$\\
	& \multicolumn{2}{l}{(also holds for real-symmetric A$\in\mathbb{R}^{nxn}$)}\\
\end{tabular}
\end{mainbox}
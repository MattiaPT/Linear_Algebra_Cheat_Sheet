\documentclass[a4paper,10pt]{article}
\usepackage[utf8]{inputenc}
\usepackage[colorlinks=true, allcolors=blue]{hyperref}
\usepackage{enumitem}
\usepackage[ngerman]{babel}
\usepackage{tabularx}
\usepackage{graphicx}
\usepackage{amssymb}
\usepackage{tikz}
\usepackage{textcomp}
\usepackage{multirow}
\usepackage{fourier}
\usepackage{makecell}


%opening
\title{
    Linear Algebra - Cheat Sheet (HS22)
}
\author{Mattia Taiana}
\date{26. December 2022}

% 3 column landscape layout with fewer margins
\usepackage[landscape, left=0.75cm, top=1cm, right=0.75cm, bottom=1cm, footskip=10pt]{geometry}
\usepackage{flowfram}
\ffvadjustfalse
\setlength{\columnsep}{0.75cm}
\Ncolumn{3}
\linespread{0.9}

% Turn off header and footer
\pagestyle{plain}

\graphicspath{ {./assets/} }

% Reduce title spacings
\usepackage{titlesec}
\titlespacing*{\section}{0pt}{8pt}{2pt}
\titlespacing*{\subsection}{0pt}{4pt}{2pt}
\titlespacing*{\subsubsection}{0pt}{3pt}{1pt}

% No auto paragraph indent
\setlength{\parindent}{0pt}

% Color boxes
\usepackage[many]{tcolorbox}
\tcbset {
  base/.style={
    bottom=0mm,
    top=0mm,
    boxrule=0mm,
    leftrule=0.5mm,
    left=1mm,
    arc=0mm, 
    fonttitle=\bfseries, 
    colbacktitle=black!5!white, 
    colback=black!2!white,
    coltitle=black,
    toptitle=0.25mm,
    bottomtitle=0mm,
    title={#1}
  }
}

\definecolor{blue}{rgb}{0.34, 0.7, 1}
\definecolor{green}{rgb}{0.2, 1, 0.4}
\definecolor{purple}{rgb}{1, 0.3, 1}

\newtcolorbox{mainbox}[2][]{
  colframe=blue,
  base={#2},
  #1
}

\newtcolorbox{howtobox}[1]{
  colframe=green, 
  base={#1}
}

\newtcolorbox{bspbox}[1]{
  colframe=purple, 
  base={#1}
}

\newtcolorbox{exbox}[1]{
  colframe=pink,
  base={#1},
  colbacktitle=black!1!white,
  colback=black!0!white,
  breakable,
}

\newtcolorbox{subbox}[2][]{
  colframe=black!20!white,
  base={#2},
  #1
}


% Mathematical typesetting & symbols
\usepackage{amsthm, mathtools, amssymb, xfrac} 
\usepackage{bm}
\usepackage{marvosym, wasysym}
\def\N{\mathbb{N}}
\def\Z{\mathbb{Z}}
\def\Q{\mathbb{Q}}
\def\R{\mathbb{R}}
\def\C{\mathbb{C}}
\newcommand*\dif{\mathop{}\!\mathrm{d}}

\newcommand{\defeq}{\vcentcolon=}
\newcommand{\eqdef}{=\vcentcolon}

\DeclareMathOperator\arccot{arccot}
\DeclareMathOperator\arctanh{arctanh}
\DeclareMathOperator\arcsinh{arcsinh}
\DeclareMathOperator\arccosh{arccosh}

\newenvironment{mcenv}{\setlist[itemize]{noitemsep,leftmargin=*}}{}

\newcommand{\axioms}[2][]{#2}
\newcommand{\definition}[2][]{#2}
\newcommand{\lemma}[2][]{#2}
\newcommand{\theorem}[2][]{#2}
\newcommand{\corollary}[2][]{#2}
\newcommand{\howto}[2][]{#2}
\newcommand{\example}[2][]{#2}
\newcommand{\mcquestion}[2]{
 \begin{exbox}{#1}
  #2
 \end{exbox}
}
\newcommand{\exercise}[2]{
 \begin{exbox}{#1}
  #2
 \end{exbox}
}
\newcommand{\smallpara}{
  \vspace{-10pt}
  \paragraph{}
}
\newcommand{\compconj}[1]{%
  \overline{#1}%
}
\newcommand{\RN}[1]{%
  \textup{\uppercase\expandafter{\romannumeral#1}}%
}
\newcommand*\circled[1]{\tikz[baseline=(char.base)]{
            \node[shape=circle,draw,inner sep=0.5pt] (char) {#1};}}
\newcommand{\ontoptext}[3][0pt]{%
	\begin{tabular}[b]{@{}c@{}}
	#2\\[#1]
	#3
	\end{tabular}}
\newcommand{\tabitem}{~~\quad\llap{\textbullet}~~}
\newcommand{\rvline}{\hspace*{-\arraycolsep}\vline\hspace*{-\arraycolsep}}



\begin{document}
 \maketitle
 \section{Complex Numbers}

\begin{mainbox}{General}

    $z = \underbrace{x}_\text{\textcolor{red}{Re}} +\:i\underbrace{y}_\text{\textcolor{red}{Im}} = \underbrace{r \cdot (\cos(\varphi) + i \cdot \sin(\varphi)) = r \cdot e^{i\varphi}}_\text{\textcolor{red}{Polarform}}\\
    \compconj{z} = x - iy = r \cdot e^{i(2\pi - \varphi)}\\
    |z| = r = \sqrt{x^{2} + y^{2}} = \sqrt{z \cdot \compconj{z}}\\
    \varphi = \begin{cases}
        arc tan(\frac{y}{x}), &\RN{1}\:Q.\\
        arc tan(\frac{y}{x}) + \pi, &\RN{2}/\RN{3}\:Q.\\
        arc tan(\frac{y}{x}) + 2\pi, &\RN{4}\:Q.\\
    \end{cases}$

\end{mainbox}
\begin{mainbox}{Operations}
\setlength{\tabcolsep}{2pt}
    \begin{tabular}{rl}
        $+/-:$ & $(x_{1} + x_{2}) + (y_{1} + y_{2})i$ \\
        $z_1\cdot z_2:$ & $(x_{1} + y_{1}i)(x_{2} + y_{2}i) = r_{1}\cdot r_{2}\cdot e^{i(\varphi_{1} + \varphi_{2})}$ \\
        $\frac{z_{1}}{z_{2}}:$ & $\frac{z_1\cdot\compconj{z}_2}{|z_2|^2} = \frac{r_{1}}{r_{2}}\cdot e^{i(\varphi_{1} - \varphi_{2})}$ \\
        $z^n:$ & $r^n\cdot e^{i\varphi n}$ \\
        $\sqrt{a}:$ & $a = z^n  \Leftrightarrow |a|\cdot e^{i\alpha} = r^n\cdot e^{i\varphi n} \underset{k = 0,...,n-1}{\begin{cases}
            r = \sqrt[n]{|a|} \\
            \varphi = \frac{\alpha + 2k\pi}{n}
        \end{cases}}$
    \end{tabular}
\end{mainbox}
\begin{mainbox}{Polynomials}
\setlength{\tabcolsep}{2pt}
    \begin{tabular}{rl}
        degree 2: & $z = \frac{b\pm \sqrt{b^2 - 4ac}}{2a}$\\
        special case: & $az^n + c = 0 \Leftrightarrow z = \sqrt[n]{-\frac{c}{a}}$
    \end{tabular}    
\end{mainbox}
\begin{mainbox}
    \text{With polynomials with complex roots, the roots occur as a complex-conjugate pair.}
    \smallskip\\
    \text{Polynomials over} $\mathbb{C}$ \text{with an odd degree}\\
    \text{have at least one root in} $\mathbb{R}$.
\end{mainbox}
\begin{tikzpicture}[scale=3,cap=round,>=latex]
        % draw the coordinates
        \draw[->] (-1.5cm,0cm) -- (1.5cm,0cm) node[right,fill=white] {$x$};
        \draw[->] (0cm,-1.5cm) -- (0cm,1.5cm) node[above,fill=white] {$y$};

        % draw the unit circle
        \draw[thick] (0cm,0cm) circle(1cm);

        \foreach \x in {0,30,...,360} {
                % lines from center to point
                \draw[gray] (0cm,0cm) -- (\x:1.2cm);
                % dots at each point
                \filldraw[black] (\x:1cm) circle(0.4pt);
                % draw each angle in degrees
                \draw (\x:0.5cm) node[fill=white] {$\x^\circ$};
        }

        % draw each angle in radians
        \foreach \x/\xtext in {
            30/\frac{\pi}{6},
            45/\frac{\pi}{4},
            60/\frac{\pi}{3},
            90/\frac{\pi}{2},
            120/\frac{2\pi}{3},
            135/\frac{3\pi}{4},
            150/\frac{5\pi}{6},
            180/\pi,
            210/\frac{7\pi}{6},
            225/\frac{5\pi}{4},
            240/\frac{4\pi}{3},
            270/\frac{3\pi}{2},
            300/\frac{5\pi}{3},
            315/\frac{7\pi}{4},
            330/\frac{11\pi}{6},
            360/2\pi}
                \draw (\x:0.8cm) node[fill=white] {$\xtext$};

        \foreach \x/\xtext/\y in {
            % the coordinates for the first quadrant
            30/\frac{\sqrt{3}}{2}/\frac{1}{2},
            45/\frac{\sqrt{2}}{2}/\frac{\sqrt{2}}{2},
            60/\frac{1}{2}/\frac{\sqrt{3}}{2},
            % the coordinates for the second quadrant
            150/-\frac{\sqrt{3}}{2}/\frac{1}{2},
            135/-\frac{\sqrt{2}}{2}/\frac{\sqrt{2}}{2},
            120/-\frac{1}{2}/\frac{\sqrt{3}}{2},
            % the coordinates for the third quadrant
            210/-\frac{\sqrt{3}}{2}/-\frac{1}{2},
            225/-\frac{\sqrt{2}}{2}/-\frac{\sqrt{2}}{2},
            240/-\frac{1}{2}/-\frac{\sqrt{3}}{2},
            % the coordinates for the fourth quadrant
            330/\frac{\sqrt{3}}{2}/-\frac{1}{2},
            315/\frac{\sqrt{2}}{2}/-\frac{\sqrt{2}}{2},
            300/\frac{1}{2}/-\frac{\sqrt{3}}{2}}
                \draw (\x:1.5cm) node[fill=white] {$\left(\xtext,\y\right)$};

        % draw the horizontal and vertical coordinates
        % the placement is better this way
        \draw (-1.25cm,0cm) node[above=1pt] {$(-1,0)$}
              (1.25cm,0cm)  node[above=1pt] {$(1,0)$}
              (0cm,-1.25cm) node[fill=white] {$(0,-1)$}
              (0cm,1.25cm)  node[fill=white] {$(0,1)$};
\end{tikzpicture}



 \section{LSE}
\begin{subbox}{Gauss-Algorithm}
	runtime: $O(n^3)$\\
	elementary row operations:\\switch, multiply and add/subtract rows\\
	goal: row echelon form\\
    \begin{tabular}{ c c c c | c l}
        $x_1$ & $x_2$ & ... & $x_n$ & & \\
        \cline{1-5}
        \circled{$a_{11}$} & $a_{12}$ & ... & $a_{1n}$ & $b_1$ & \multirow{5}{*}{\hspace{-1em}$\left.\begin{array}{l}
                \\
                \\
                \\
                \\
                \\
                \end{array}\right\rbrace \text{rank}$} \\
        0 & \circled{$a_{22}$} & ... & $a_{2n}$ & $b_2$ & \\
        ... & ... & ... & ... & ... & \\
        0 & 0 & 0 & \circled{$a_{rn}$} & $b_r$ & \\
        ... & ... & ... & ... & ... & \multirow{2}{*}{\hspace{-1em}$\left.\begin{array}{l}
                \\
                \\
                \end{array}\right\rbrace $\ontoptext[-4pt]{consistency}{conditions}} \\
        0 & 0 & ... & 0 & $b_m$ & \\
        \multicolumn{4}{c}{\upbracefill} & \multicolumn{1}{c}{\upbracefill}\\
        \multicolumn{4}{c}{LHS}& \multicolumn{1}{c}{RHS}\\
    \end{tabular}\\
    if RHS always 0: homogeneous LSE\\
	consistency conditions (VB): $b_{r+1} = ... = b_m = 0$
\end{subbox}

\begin{mainbox}{Number of solutions}
\setlength{\tabcolsep}{2pt}
\begin{tabular}{rl}
	\textcolor{blue}{T 1.1} & $Ax = b$ min. 1 solution $\Leftrightarrow (r = m)$ or $(r < m + VB)$\\
	& if this is the case:\\
	& \tabitem $r = n$: only 1 solution\\
	& \tabitem $r < n$: $\infty$ many solutions\\
	& $\Rightarrow r = m$:\\
	& \tabitem $r = n = m$: only 1 solution, \textcolor{red}{non-singular} LSE\\
	& \tabitem $r < n$: $\infty$ many solutions, $(n-r)$ free parameters\\
	& $\Rightarrow r < m$:\\
	& \tabitem $r = n$: only 1 solution\\
	& \tabitem $r < n$: $\infty$ many solutions, $(n-r)$ free parameters\\
	\rule{0pt}{3ex}
	K 1.5 & In a homogeneous LSE non-trivial solutions exist\\
	& only if $r < n$.\\
	\rule{0pt}{3ex}
	\textcolor{blue}{T 2.5} & $Ax = b$ has a solution $\Leftrightarrow$\\
	& b is a linear combination of columnvectors of A\\
\end{tabular}\\
\end{mainbox}
 \section{Matrices \& Vectors}
\begin{mainbox}{General}
$m$x$n$ Matrices have m rows and n columns.\\
The element $(i,j)$ can be denoted as $a_{i,j}$ or $(A)_{i,j}$
\smallskip\\
\setlength{\tabcolsep}{2pt}
\begin{tabular}{rll}
	\textcolor{blue}{T 2.1} & \multicolumn{1}{l|}{$(\alpha\beta)A = \alpha(\beta A)$} & $(A+B)+C = A+(B+C)$ \\
	& \multicolumn{1}{l|}{$(\alpha A)B = \alpha(AB)$} & $(AB)\cdot C = A\cdot (BC)$ \\
	& \multicolumn{1}{l|}{$(\alpha + \beta)\cdot A = \alpha A + \beta A$} & $(A+B)\cdot C = AC + BC$ \\
	& \multicolumn{1}{l|}{$\alpha (A+B) = \alpha A + \alpha B$} & $A\cdot (B+C) = AB+AC$ \\
	& \multicolumn{1}{l|}{$A+B = B+A$} & \\
	\multicolumn{3}{c}{\danger in general$AB \neq BA$}\\
	\multicolumn{3}{c}{If $AB = BA$ we say «A and B commute»}\\
	\rule{0pt}{3ex}
	Def. & \multicolumn{2}{l}{If $AB = O$, we call $A$ and $B$ \textcolor{blue}{divisors of zero}.}\\
	\rule{0pt}{3ex}
	Def. & \multicolumn{2}{l}{A \textcolor{blue}{linear combination} of vectors $a_1 ... a_n$ is an}\\
 	& \multicolumn{2}{l}{expression of the following type:}\\
 	& \multicolumn{2}{l}{$\alpha_n\cdot a_n + ... + \alpha_1\cdot a_1$}\\
 	\rule{0pt}{3ex}
	Def. & \multicolumn{2}{l}{A matrix is \textcolor{blue}{symmetric} when $A^T = A$ and}\\
	& \multicolumn{2}{l}{\textcolor{blue}{Hermitian} when $A^H = A$ (real diagonal).}\\
	\rule{0pt}{3ex}
	Def. & \multicolumn{2}{l}{A matrix is \textcolor{blue}{skew-symmetric} when $A^T = -A$.}\\
	& \multicolumn{2}{l}{(zeros on diagonal)}\\
	\rule{0pt}{3ex}
	\textcolor{blue}{T 2.6} & \multicolumn{2}{l}{$(A^T)^T = A$ \qquad\qquad $\:(\alpha A)^T = \alpha(A^T)$}\\
	& \multicolumn{2}{l}{$(AB)^T = B^TA^T$ \qquad $(A+B)^T = A^T + B^T$}\\
	\rule{0pt}{3ex}
	\textcolor{blue}{T 2.7} & \multicolumn{2}{l}{If A, B symmetric: $AB = BA \Leftrightarrow AB$ symmetric}\\
	& \multicolumn{2}{l}{For any A: $A^TA = AA^T$ (symmetric)}\\
\end{tabular}
\end{mainbox}

\begin{mainbox}{Scalar Product and Norm}
\setlength{\tabcolsep}{2pt}
\begin{tabular}{rll}
	Def. & \textcolor{red}{Eucl. scalar product} (SP):& $\langle x,y\rangle :\equiv x^Hy$\\
	& (inner product) & \\
	\rule{0pt}{3ex}
	\textcolor{blue}{T 2.9} & linearity in second factor: & $\langle x,y+z\rangle = \langle x,y\rangle + \langle x,z\rangle$\\
	& & $\langle x,\alpha y\rangle = \alpha \langle x,y\rangle$\\
	& symmetric / hermitian: & $\langle x,y\rangle = \compconj{\langle y,x \rangle}$\\
	& positive definite: & $\langle x,x\rangle \geq 0$; if $\;'='\;\Rightarrow\;x = 0$\\ 
	\rule{0pt}{3ex}
	\textcolor{blue}{C 2.10} & bilinearity in $\mathbb{R}^n$: & $\langle w+x,y\rangle = \langle w,y\rangle + \langle x,y\rangle$\\
	& & $\langle \alpha x,y\rangle = \alpha \langle x,y\rangle$\\
	& sesquilinearity in $\mathbb{C}^n$: & $\langle w+x,y\rangle = \langle w,y\rangle + \langle x,y\rangle$\\
	& & $\langle \alpha x,y\rangle = \compconj{\alpha} \langle x,y\rangle$\\ 
	\rule{0pt}{3ex}
	Def. & \textcolor{red}{Eucl. norm / 2-norm}: & $||x|| :\equiv \sqrt{\langle x,x \rangle}$\\
	\rule{0pt}{3ex}
\end{tabular}
\textcolor{blue}{(Cauchy-)Schwarz inequality (CBS inequality)} \\
\begin{tabular}{l}
	$|\langle x,y\rangle| \leq ||x|| \cdot ||y||$\\
	The equality holds iff y is a multiple of x or vice versa\\
\end{tabular}\\
\smallskip\\
\setlength{\tabcolsep}{2pt}
\begin{tabular}{rrl}
	\textcolor{blue}{T 2.12} & \multicolumn{2}{l}{The following holds for the 2-norm:} \\
	& (N1) & positive definite: $||x|| \geq 0$, if$\;'='\;\Rightarrow\; x = 0$\\
	& (N2) & $||\alpha x|| = |\alpha |\;||x||$\\
	& (N3) & \textcolor{red}{triangle inequality}: $||x\pm y|| \leq ||x|| + ||y||$\\
	\rule{0pt}{3ex}
	Def. & \multicolumn{2}{l}{angle $\varphi$ between x and y: $\varphi = arc\;cos\left(\frac{\langle x,y\rangle}{||x||\cdot ||y||}\right)$}\\
	\rule{0pt}{3ex}
	Def. & \multicolumn{2}{l}{x and y are \textcolor{blue}{orthogonal}, if $\langle x,y\rangle = 0$; $x \perp y$}\\
	\rule{0pt}{3ex}
	\textcolor{blue}{T 2.13} & \multicolumn{2}{l}{\textcolor{blue}{Pythagoras}: $||x \pm y||^2 = ||x||^2 + ||y||^2$, if $x \perp y$}\\
	\rule{0pt}{3ex}	
	Def. & \multicolumn{2}{l}{\textcolor{blue}{p-Norm}: $||x||_p :\equiv (|x_1|^p + ... + |x_n|^p)^{\frac{1}{p}}$}\\
\end{tabular}
\end{mainbox}

\begin{mainbox}{Outer Product and Projection}
\setlength{\tabcolsep}{2pt}
\begin{tabular}{rl}
	Def. & The \textcolor{blue}{outer product} is the matrix that is returned,\\
	& when multiplying the vectors x and y: $x\cdot y^H$\\
	& ($rank = 1$)\\
	\rule{0pt}{3ex}
	\textcolor{blue}{T 2.15} & The \textcolor{red}{orthogonal projection} $P_yx$ of $x$ on $y$\\
	& is given by: $P_yx :\equiv \frac{1}{||y||^2}yy^Hx$\\
	\rule{0pt}{3ex}
	Def. & The \textcolor{blue}{projection matrix} $P_y = \frac{1}{||y||^2}\cdot yy^H$\\
	& $P_y^H = P_y$ (Hermitian), $P_y^2 = P_y$ (Idempotent)\\
\end{tabular}

\end{mainbox}

\begin{mainbox}{Linear Transformations}
For all $x, \tilde{x} \in \mathbb{E}^n$ and $\gamma \in \mathbb{E}$:\\
$A(\gamma x + \tilde{x}) = \gamma (Ax) + (A\tilde{x})$
\smallskip\\
Def. \textcolor{red}{image of A}: $\;imA\;:\equiv\;\{Ax\in \mathbb{E}^m; x\in \mathbb{E}^n\}$\\
\end{mainbox}

\begin{mainbox}{Inverse}
\setlength{\tabcolsep}{2pt}
\begin{tabular}{rl}
	Def. & A nxn matrix A is \textcolor{blue}{invertible}, if there exists\\
	& a matrix $A^{-1}$, such that $A\cdot A^{-1} = I_n = A^{-1}A$.\\
	\rule{0pt}{3ex}
	\textcolor{blue}{T 2.17} & 4 equivalent statements:\\
	i) & A is invertible\\
	ii) & $\exists X$ such that $AX = I_n$\\
	iii) & X is definitive\\
	iv) & A is non-singular, i.e. rank A = n\\
	\rule{0pt}{3ex}
	\textcolor{blue}{T 2.18} & With two non-singular nxn matrices $A$ and $B$:\\
	i) & $A^{-1}$ is non-singular and $(A^{-1})^{-1} = A$\\
	ii) & $AB$ is non-singular and $(AB)^{-1} = B^{-1}A^{-1}$\\
	iii) & $A^H$ is non-singular and $(A^H)^{-1} = (A^{-1})^H$\\
	\rule{0pt}{3ex}
	\textcolor{blue}{T 2.19} & If $A$ is non-singular, $Ax = b$ has exactly one\\
	& solution for every b: $x = A^{-1}b$\\
\end{tabular}\\
\smallskip\\
\textcolor{red}{Find inverse $O(n^3)$}: $[\;A\;|\;I\;]\longrightarrow[\;I\;|\;A^{-1}\;]$\\
-> using elementary row operations\\
\smallskip\\
$A = \begin{bmatrix}
a & b \\
c & d \\
\end{bmatrix}$ and $detA \neq 0\;\Leftrightarrow A^{-1}=\frac{1}{ad-bc}\cdot \begin{bmatrix}
d & -b \\
-c & a \\
\end{bmatrix}$
\end{mainbox}

\begin{mainbox}{Orthogonal and Unitary Matrices}
\setlength{\tabcolsep}{2pt}
\begin{tabular}{rl}
	Def. & We call a matrix unitary or orthogonal, \\
	& if $A^HA = I_n$, $A^TA = I_n$ respectively.\\
	\rule{0pt}{3ex}
	\textcolor{blue}{T 2.20} & Let $A$ and $B$ be unitary:\\
	i) & $A$ is non-singular and $A^{-1} = A^H$\\
	ii) & $AA^H = I_n$\\
	iii) & $A^{-1}$ is unitary (/orthogonal)\\
	iv) & $AB$ is unitary (/orthogonal)\\
	\rule{0pt}{3ex}
	\textcolor{blue}{T 2.21} & A linear transformation defined by an orthog-\\
	& onal or unitary nxn matrix A is \textcolor{red}{length preserving} \\
	& (/isometric) and \textcolor{red}{angle preserving}:\\
	& $||Ax|| = ||x||$, $\langle Ax,Ay\rangle = \langle x,y\rangle$\\
\end{tabular}
\end{mainbox}

\begin{mainbox}{Examples of Important Matrices}
\textcolor{blue}{Rotationmatrices (orthogonal)}\\
$\begin{bmatrix}
cos(\varphi) & sin(\varphi) \\
-sin(\varphi) & cos(\varphi) \\
\end{bmatrix}$ or $\begin{bmatrix}
cos(\varphi) & 0 & sin(\varphi) & 0 \\
0 & 1 & 0 & 0 \\
-sin(\varphi) & 0 & cos(\varphi) & 0 \\
0 & 0 & 0 & 1 \\
\end{bmatrix}$
\\\smallskip\\
\textcolor{blue}{Permutationmatrices (orthogonal)}\\
$\begin{bmatrix}
1 & 0 & 0 \\
0 & 1 & 0 \\
0 & 0 & 1 \\
\end{bmatrix}$ or
$\begin{bmatrix}
0 & 1 & 0 & 0 \\
0 & 0 & 1 & 0 \\
0 & 0 & 0 & 1 \\
1 & 0 & 0 & 0 \\
\end{bmatrix}$
\\\smallskip\\
\textcolor{blue}{Blockmatrices}\\
$A = \left(\begin{array}{@{}c|c@{}}
  a_{11} & a_{12} \\
\hline
  a_{21} & a_{22}
\end{array}\right)$, if invertible $A^{-1} = \left(\begin{array}{@{}c|c@{}}
  a_{11}^{-1} & a_{12}^{-1} \\
\hline
  a_{21}^{-1} & a_{22}^{-1}
\end{array}\right)$
\end{mainbox}
 \section{LU-Decomposition}
\begin{howtobox}{LU-Decomposition}
The \textcolor{red}{LU-Decomposition} is a tool to\\
solve SLE. It does this by factorizing a matrix, making\\
it easy to solve the same matrix vor different RHS.\\
\begin{tabular}{l|l}
	1. Find $PA = LR$ & $\left[\arraycolsep=3.8pt\begin{array}{ccc|ccc|ccc}
	1 & 0 & 0 & 2 & 1 & 2 & 1 & 0 & 0\\
	0 & 1 & 0 & 1 & 2 & 3 & 0 & 1 & 0\\
	0 & 0 & 1 & 2 & 2 & 2 & 0 & 0 & 1\\
	\end{array}\right] $\\
	\makecell[l]{2. Solve $Lc = Pb$\\\quad(forward subst.)} & $\left[\arraycolsep=3.64pt\begin{array}{ccc|ccc|ccc}
	1 & 0 & 0 & 2 & 1 & 2 & 1 & 0 & 0\\
	0 & 1 & 0 & 0 & \frac{3}{2} & 2 & \textcolor{red}{\frac{1}{2}} & 1 & 0\\
	0 & 0 & 1 & 0 & 1 & 0 & \textcolor{red}{1} & 0 & 1\\
	\end{array}\right] $\\
	\makecell[l]{3. Solve $Rx = c$\\\quad(backward subst.)} & $\left[\arraycolsep=3pt\begin{array}{ccc|ccc|ccc}
	1 & 0 & 0 & 2 & 1 & 2 & 1 & 0 & 0\\
	0 & 1 & 0 & 0 & \frac{3}{2} & 2 & \textcolor{red}{\frac{1}{2}} & 1 & 0\\
	\undermat{P}{0 & 0 & 1} & \undermat{R}{0 & 0 & -\frac{4}{3}} & \undermat{L}{\textcolor{red}{1} & \textcolor{red}{\frac{2}{3}} & 1}\\
	\end{array}\right] $\\
	& \\
	\makecell[l]{If rows are swapped\\ P gets permutated.} & \\
	& \\
	\multicolumn{2}{l}{\makecell[l]{\textcolor{red}{partial pivoting} as a \textcolor{red}{pivot}\\\textcolor{red}{strategy} to minimize rounding errors}}\\
\end{tabular}
\end{howtobox}
\end{document}

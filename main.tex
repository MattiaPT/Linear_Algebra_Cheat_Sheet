\documentclass[a4paper,10pt]{article}
\usepackage[utf8]{inputenc}
\usepackage[colorlinks=true, allcolors=blue]{hyperref}
\usepackage{enumitem}
\usepackage[ngerman]{babel}
\usepackage{tabularx}
\usepackage{graphicx}
\usepackage{amssymb}
\usepackage{tikz}
\usepackage{textcomp}
\usepackage{multirow}
\usepackage{fourier}
\usepackage{makecell}
\usepackage{tikz-cd}
\usetikzlibrary{arrows,positioning}
\tikzset{
  shift left/.style ={commutative diagrams/shift left={#1}},
  shift right/.style={commutative diagrams/shift right={#1}}
}


%opening
\title{
    Linear Algebra - Cheat Sheet (HS22)
}
\author{Mattia Taiana}
\date{26. December 2022}

% 3 column landscape layout with fewer margins
\usepackage[landscape, left=0.75cm, top=1cm, right=0.75cm, bottom=1cm, footskip=10pt]{geometry}
\usepackage{flowfram}
\ffvadjustfalse
\setlength{\columnsep}{0.75cm}
\Ncolumn{3}
\linespread{0.9}

% Turn off header and footer
\pagestyle{plain}

\graphicspath{ {./assets/} }

% Reduce title spacings
\usepackage{titlesec}
\titlespacing*{\section}{0pt}{8pt}{2pt}
\titlespacing*{\subsection}{0pt}{4pt}{2pt}
\titlespacing*{\subsubsection}{0pt}{3pt}{1pt}

% No auto paragraph indent
\setlength{\parindent}{0pt}

% Color boxes
\usepackage[many]{tcolorbox}
\tcbset {
  base/.style={
    bottom=0mm,
    top=0mm,
    boxrule=0mm,
    leftrule=0.5mm,
    left=1mm,
    arc=0mm, 
    fonttitle=\bfseries, 
    colbacktitle=black!5!white, 
    colback=black!2!white,
    coltitle=black,
    toptitle=0.25mm,
    bottomtitle=0mm,
    title={#1}
  }
}

\definecolor{blue}{rgb}{0.34, 0.7, 1}
\definecolor{green}{rgb}{0.2, 1, 0.4}
\definecolor{purple}{rgb}{1, 0.3, 1}

\newtcolorbox{mainbox}[2][]{
  colframe=blue,
  base={#2},
  #1
}

\newtcolorbox{howtobox}[1]{
  colframe=green, 
  base={#1}
}

\newtcolorbox{bspbox}[1]{
  colframe=purple, 
  base={#1}
}

\newtcolorbox{exbox}[1]{
  colframe=pink,
  base={#1},
  colbacktitle=black!1!white,
  colback=black!0!white,
  breakable,
}

\newtcolorbox{subbox}[2][]{
  colframe=black!20!white,
  base={#2},
  #1
}


% Mathematical typesetting & symbols
\usepackage{amsthm, mathtools, amssymb, xfrac} 
\usepackage{bm}
\usepackage{marvosym, wasysym}
\def\N{\mathbb{N}}
\def\Z{\mathbb{Z}}
\def\Q{\mathbb{Q}}
\def\R{\mathbb{R}}
\def\C{\mathbb{C}}
\newcommand*\dif{\mathop{}\!\mathrm{d}}

\newcommand{\defeq}{\vcentcolon=}
\newcommand{\eqdef}{=\vcentcolon}

\DeclareMathOperator\arccot{arccot}
\DeclareMathOperator\arctanh{arctanh}
\DeclareMathOperator\arcsinh{arcsinh}
\DeclareMathOperator\arccosh{arccosh}

\newenvironment{mcenv}{\setlist[itemize]{noitemsep,leftmargin=*}}{}

\newcommand{\axioms}[2][]{#2}
\newcommand{\definition}[2][]{#2}
\newcommand{\lemma}[2][]{#2}
\newcommand{\theorem}[2][]{#2}
\newcommand{\corollary}[2][]{#2}
\newcommand{\howto}[2][]{#2}
\newcommand{\example}[2][]{#2}
\newcommand{\mcquestion}[2]{
 \begin{exbox}{#1}
  #2
 \end{exbox}
}
\newcommand{\exercise}[2]{
 \begin{exbox}{#1}
  #2
 \end{exbox}
}
\newcommand{\smallpara}{
  \vspace{-10pt}
  \paragraph{}
}
\newcommand{\compconj}[1]{%
  \overline{#1}%
}
\newcommand{\RN}[1]{%
  \textup{\uppercase\expandafter{\romannumeral#1}}%
}
\newcommand*\circled[1]{\tikz[baseline=(char.base)]{
            \node[shape=circle,draw,inner sep=0.5pt] (char) {#1};}}
\newcommand{\ontoptext}[3][0pt]{%
	\begin{tabular}[b]{@{}c@{}}
	#2\\[#1]
	#3
	\end{tabular}}
\newcommand{\tabitem}{~~\quad\llap{\textbullet}~~}
\newcommand{\rvline}{\hspace*{-\arraycolsep}\vline\hspace*{-\arraycolsep}}
\newcommand\undermat[2]{%
  \makebox[0pt][l]{$\smash{\underbrace{\phantom{%
    \begin{matrix}#2\end{matrix}}}_{\text{$#1$}}}$}#2}



\begin{document}
 \maketitle
 \section{Complex Numbers}

\begin{mainbox}{General}

    $z = \underbrace{x}_\text{\textcolor{red}{Re}} +\:i\underbrace{y}_\text{\textcolor{red}{Im}} = \underbrace{r \cdot (\cos(\varphi) + i \cdot \sin(\varphi)) = r \cdot e^{i\varphi}}_\text{\textcolor{red}{Polarform}}\\
    \compconj{z} = x - iy = r \cdot e^{i(2\pi - \varphi)}\\
    |z| = r = \sqrt{x^{2} + y^{2}} = \sqrt{z \cdot \compconj{z}}\\
    \varphi = \begin{cases}
        arc tan(\frac{y}{x}), &\RN{1}\:Q.\\
        arc tan(\frac{y}{x}) + \pi, &\RN{2}/\RN{3}\:Q.\\
        arc tan(\frac{y}{x}) + 2\pi, &\RN{4}\:Q.\\
    \end{cases}$

\end{mainbox}
\begin{mainbox}{Operations}
\setlength{\tabcolsep}{2pt}
    \begin{tabular}{rl}
        $+/-:$ & $(x_{1} + x_{2}) + (y_{1} + y_{2})i$ \\
        $z_1\cdot z_2:$ & $(x_{1} + y_{1}i)(x_{2} + y_{2}i) = r_{1}\cdot r_{2}\cdot e^{i(\varphi_{1} + \varphi_{2})}$ \\
        $\frac{z_{1}}{z_{2}}:$ & $\frac{z_1\cdot\compconj{z}_2}{|z_2|^2} = \frac{r_{1}}{r_{2}}\cdot e^{i(\varphi_{1} - \varphi_{2})}$ \\
        $z^n:$ & $r^n\cdot e^{i\varphi n}$ \\
        $\sqrt{a}:$ & $a = z^n  \Leftrightarrow |a|\cdot e^{i\alpha} = r^n\cdot e^{i\varphi n} \underset{k = 0,...,n-1}{\begin{cases}
            r = \sqrt[n]{|a|} \\
            \varphi = \frac{\alpha + 2k\pi}{n}
        \end{cases}}$
    \end{tabular}
\end{mainbox}
\begin{mainbox}{Polynomials}
\setlength{\tabcolsep}{2pt}
    \begin{tabular}{rl}
        degree 2: & $z = \frac{b\pm \sqrt{b^2 - 4ac}}{2a}$\\
        special case: & $az^n + c = 0 \Leftrightarrow z = \sqrt[n]{-\frac{c}{a}}$
    \end{tabular}    
\end{mainbox}
\begin{mainbox}
    \text{With polynomials with complex roots, the roots occur as a complex-conjugate pair.}
    \smallskip\\
    \text{Polynomials over} $\mathbb{C}$ \text{with an odd degree}\\
    \text{have at least one root in} $\mathbb{R}$.
\end{mainbox}
\input{chapters/unit_circle.tex}


 \section{LSE}
\begin{mainbox}{Gauss-Algorithm}
	runtime: $O(n^3)$\\
	valid Ops: switch, multiply and add/subtract rows\\
	goal: row echelon form\\
    \begin{tabular}{ c c c c | c l}
        $x_1$ & $x_2$ & ... & $x_n$ & & \\
        \cline{1-5}
        \circled{$a_{11}$} & $a_{12}$ & ... & $a_{1n}$ & $b_1$ & \multirow{5}{*}{\hspace{-1em}$\left.\begin{array}{l}
                \\
                \\
                \\
                \\
                \\
                \end{array}\right\rbrace \text{rank r}$} \\
        0 & \circled{$a_{22}$} & ... & $a_{2n}$ & $b_2$ & \\
        ... & ... & ... & ... & ... & \\
        0 & 0 & 0 & \circled{$a_{rn}$} & $b_r$ & \\
        ... & ... & ... & ... & ... & \multirow{2}{*}{\hspace{-1em}$\left.\begin{array}{l}
                \\
                \\
                \end{array}\right\rbrace $\ontoptext[-4pt]{consistency}{conditions}} \\
        0 & 0 & ... & 0 & $b_m$ & \\
        \multicolumn{4}{c}{\upbracefill} & \multicolumn{1}{c}{\upbracefill}\\
        \multicolumn{4}{c}{LHS}& \multicolumn{1}{c}{RHS}\\
    \end{tabular}\\
    if RHS always 0: homogeneous LSE\\
	consistency conditions: $b_{r+1} = ... = b_m = 0$
\end{mainbox}
 \section{Matrices \& Vectors}
\begin{mainbox}{General}
$m$x$n$ Matrices have m rows and n columns.\\
The element $(i,j)$ can be denoted as $a_{i,j}$ or $(A)_{i,j}$
\smallskip\\
\setlength{\tabcolsep}{2pt}
\begin{tabular}{rll}
	\textcolor{blue}{T 2.1} & \multicolumn{1}{l|}{$(\alpha\beta)A = \alpha(\beta A)$} & $(A+B)+C = A+(B+C)$ \\
	& \multicolumn{1}{l|}{$(\alpha A)B = \alpha(AB)$} & $(AB)\cdot C = A\cdot (BC)$ \\
	& \multicolumn{1}{l|}{$(\alpha + \beta)\cdot A = \alpha A + \beta A$} & $(A+B)\cdot C = AC + BC$ \\
	& \multicolumn{1}{l|}{$\alpha (A+B) = \alpha A + \alpha B$} & $A\cdot (B+C) = AB+AC$ \\
	& \multicolumn{1}{l|}{$A+B = B+A$} & \\
	\multicolumn{3}{c}{\danger in general$AB \neq BA$}\\
	\multicolumn{3}{c}{If $AB = BA$ we say «A and B commute»}\\
	\rule{0pt}{3ex}
	Def. & \multicolumn{2}{l}{If $AB = O$, we call $A$ and $B$ \textcolor{blue}{divisors of zero}.}\\
	\rule{0pt}{3ex}
	Def. & \multicolumn{2}{l}{A \textcolor{blue}{linear combination} of vectors $a_1 ... a_n$ is an}\\
 	& \multicolumn{2}{l}{expression of the following type:}\\
 	& \multicolumn{2}{l}{$\alpha_n\cdot a_n + ... + \alpha_1\cdot a_1$}\\
 	\rule{0pt}{3ex}
	Def. & \multicolumn{2}{l}{A matrix is \textcolor{blue}{symmetric} when $A^T = A$ and}\\
	& \multicolumn{2}{l}{\textcolor{blue}{Hermitian} when $A^H = A$ (real diagonal).}\\
	\rule{0pt}{3ex}
	Def. & \multicolumn{2}{l}{A matrix is \textcolor{blue}{skew-symmetric} when $A^T = -A$.}\\
	& \multicolumn{2}{l}{(zeros on diagonal)}\\
	\rule{0pt}{3ex}
	\textcolor{blue}{T 2.6} & \multicolumn{2}{l}{$(A^T)^T = A$ \qquad\qquad $\:(\alpha A)^T = \alpha(A^T)$}\\
	& \multicolumn{2}{l}{$(AB)^T = B^TA^T$ \qquad $(A+B)^T = A^T + B^T$}\\
	\rule{0pt}{3ex}
	\textcolor{blue}{T 2.7} & \multicolumn{2}{l}{If A, B symmetric: $AB = BA \Leftrightarrow AB$ symmetric}\\
	& \multicolumn{2}{l}{For any A: $A^TA = AA^T$ (symmetric)}\\
\end{tabular}
\end{mainbox}

\begin{mainbox}{Scalar Product and Norm}
\setlength{\tabcolsep}{2pt}
\begin{tabular}{rll}
	Def. & \textcolor{red}{Eucl. scalar product} (SP):& $\langle x,y\rangle :\equiv x^Hy$\\
	& (inner product) & \\
	\rule{0pt}{3ex}
	\textcolor{blue}{T 2.9} & linearity in second factor: & $\langle x,y+z\rangle = \langle x,y\rangle + \langle x,z\rangle$\\
	& & $\langle x,\alpha y\rangle = \alpha \langle x,y\rangle$\\
	& symmetric / hermitian: & $\langle x,y\rangle = \compconj{\langle y,x \rangle}$\\
	& positive definite: & $\langle x,x\rangle \geq 0$; if $\;'='\;\Rightarrow\;x = 0$\\ 
	\rule{0pt}{3ex}
	\textcolor{blue}{C 2.10} & bilinearity in $\mathbb{R}^n$: & $\langle w+x,y\rangle = \langle w,y\rangle + \langle x,y\rangle$\\
	& & $\langle \alpha x,y\rangle = \alpha \langle x,y\rangle$\\
	& sesquilinearity in $\mathbb{C}^n$: & $\langle w+x,y\rangle = \langle w,y\rangle + \langle x,y\rangle$\\
	& & $\langle \alpha x,y\rangle = \compconj{\alpha} \langle x,y\rangle$\\ 
	\rule{0pt}{3ex}
	Def. & \textcolor{red}{Eucl. norm / 2-norm}: & $||x|| :\equiv \sqrt{\langle x,x \rangle}$\\
	\rule{0pt}{3ex}
\end{tabular}
\textcolor{blue}{(Cauchy-)Schwarz inequality (CBS inequality)} \\
\begin{tabular}{l}
	$|\langle x,y\rangle| \leq ||x|| \cdot ||y||$\\
	The equality holds iff y is a multiple of x or vice versa\\
\end{tabular}\\
\smallskip\\
\setlength{\tabcolsep}{2pt}
\begin{tabular}{rrl}
	\textcolor{blue}{T 2.12} & \multicolumn{2}{l}{The following holds for the 2-norm:} \\
	& (N1) & positive definite: $||x|| \geq 0$, if$\;'='\;\Rightarrow\; x = 0$\\
	& (N2) & $||\alpha x|| = |\alpha |\;||x||$\\
	& (N3) & \textcolor{red}{triangle inequality}: $||x\pm y|| \leq ||x|| + ||y||$\\
	\rule{0pt}{3ex}
	Def. & \multicolumn{2}{l}{angle $\varphi$ between x and y: $\varphi = arc\;cos\left(\frac{\langle x,y\rangle}{||x||\cdot ||y||}\right)$}\\
	\rule{0pt}{3ex}
	Def. & \multicolumn{2}{l}{x and y are \textcolor{blue}{orthogonal}, if $\langle x,y\rangle = 0$; $x \perp y$}\\
	\rule{0pt}{3ex}
	\textcolor{blue}{T 2.13} & \multicolumn{2}{l}{\textcolor{blue}{Pythagoras}: $||x \pm y||^2 = ||x||^2 + ||y||^2$, if $x \perp y$}\\
	\rule{0pt}{3ex}	
	Def. & \multicolumn{2}{l}{\textcolor{blue}{p-Norm}: $||x||_p :\equiv (|x_1|^p + ... + |x_n|^p)^{\frac{1}{p}}$}\\
\end{tabular}
\end{mainbox}

\begin{mainbox}{Outer Product and Projection}
\setlength{\tabcolsep}{2pt}
\begin{tabular}{rl}
	Def. & The \textcolor{blue}{outer product} is the matrix that is returned,\\
	& when multiplying the vectors x and y: $x\cdot y^H$\\
	& ($rank = 1$)\\
	\rule{0pt}{3ex}
	\textcolor{blue}{T 2.15} & The \textcolor{red}{orthogonal projection} $P_yx$ of $x$ on $y$\\
	& is given by: $P_yx :\equiv \frac{1}{||y||^2}yy^Hx$\\
	\rule{0pt}{3ex}
	Def. & The \textcolor{blue}{projection matrix} $P_y = \frac{1}{||y||^2}\cdot yy^H$\\
	& $P_y^H = P_y$ (Hermitian), $P_y^2 = P_y$ (Idempotent)\\
\end{tabular}

\end{mainbox}

\begin{mainbox}{Linear Transformations}
For all $x, \tilde{x} \in \mathbb{E}^n$ and $\gamma \in \mathbb{E}$:\\
$A(\gamma x + \tilde{x}) = \gamma (Ax) + (A\tilde{x})$
\smallskip\\
Def. \textcolor{red}{image of A}: $\;imA\;:\equiv\;\{Ax\in \mathbb{E}^m; x\in \mathbb{E}^n\}$\\
\end{mainbox}

\begin{mainbox}{Inverse}
\setlength{\tabcolsep}{2pt}
\begin{tabular}{rl}
	Def. & A nxn matrix A is \textcolor{blue}{invertible}, if there exists\\
	& a matrix $A^{-1}$, such that $A\cdot A^{-1} = I_n = A^{-1}A$.\\
	\rule{0pt}{3ex}
	\textcolor{blue}{T 2.17} & 4 equivalent statements:\\
	i) & A is invertible\\
	ii) & $\exists X$ such that $AX = I_n$\\
	iii) & X is definitive\\
	iv) & A is non-singular, i.e. rank A = n\\
	\rule{0pt}{3ex}
	\textcolor{blue}{T 2.18} & With two non-singular nxn matrices $A$ and $B$:\\
	i) & $A^{-1}$ is non-singular and $(A^{-1})^{-1} = A$\\
	ii) & $AB$ is non-singular and $(AB)^{-1} = B^{-1}A^{-1}$\\
	iii) & $A^H$ is non-singular and $(A^H)^{-1} = (A^{-1})^H$\\
	\rule{0pt}{3ex}
	\textcolor{blue}{T 2.19} & If $A$ is non-singular, $Ax = b$ has exactly one\\
	& solution for every b: $x = A^{-1}b$\\
\end{tabular}\\
\smallskip\\
\textcolor{red}{Find inverse $O(n^3)$}: $[\;A\;|\;I\;]\longrightarrow[\;I\;|\;A^{-1}\;]$\\
-> using elementary row operations\\
\smallskip\\
$A = \begin{bmatrix}
a & b \\
c & d \\
\end{bmatrix}$ and $detA \neq 0\;\Leftrightarrow A^{-1}=\frac{1}{ad-bc}\cdot \begin{bmatrix}
d & -b \\
-c & a \\
\end{bmatrix}$
\end{mainbox}

\begin{mainbox}{Orthogonal and Unitary Matrices}
\setlength{\tabcolsep}{2pt}
\begin{tabular}{rl}
	Def. & We call a matrix unitary or orthogonal, \\
	& if $A^HA = I_n$, $A^TA = I_n$ respectively.\\
	\rule{0pt}{3ex}
	\textcolor{blue}{T 2.20} & Let $A$ and $B$ be unitary:\\
	i) & $A$ is non-singular and $A^{-1} = A^H$\\
	ii) & $AA^H = I_n$\\
	iii) & $A^{-1}$ is unitary (/orthogonal)\\
	iv) & $AB$ is unitary (/orthogonal)\\
	\rule{0pt}{3ex}
	\textcolor{blue}{T 2.21} & A linear transformation defined by an orthog-\\
	& onal or unitary nxn matrix A is \textcolor{red}{length preserving} \\
	& (/isometric) and \textcolor{red}{angle preserving}:\\
	& $||Ax|| = ||x||$, $\langle Ax,Ay\rangle = \langle x,y\rangle$\\
\end{tabular}
\end{mainbox}

\begin{mainbox}{Examples of Important Matrices}
\textcolor{blue}{Rotationmatrices (orthogonal)}\\
$\begin{bmatrix}
cos(\varphi) & sin(\varphi) \\
-sin(\varphi) & cos(\varphi) \\
\end{bmatrix}$ or $\begin{bmatrix}
cos(\varphi) & 0 & sin(\varphi) & 0 \\
0 & 1 & 0 & 0 \\
-sin(\varphi) & 0 & cos(\varphi) & 0 \\
0 & 0 & 0 & 1 \\
\end{bmatrix}$
\\\smallskip\\
\textcolor{blue}{Permutationmatrices (orthogonal)}\\
$\begin{bmatrix}
1 & 0 & 0 \\
0 & 1 & 0 \\
0 & 0 & 1 \\
\end{bmatrix}$ or
$\begin{bmatrix}
0 & 1 & 0 & 0 \\
0 & 0 & 1 & 0 \\
0 & 0 & 0 & 1 \\
1 & 0 & 0 & 0 \\
\end{bmatrix}$
\\\smallskip\\
\textcolor{blue}{Blockmatrices}\\
$A = \left(\begin{array}{@{}c|c@{}}
  a_{11} & a_{12} \\
\hline
  a_{21} & a_{22}
\end{array}\right)$, if invertible $A^{-1} = \left(\begin{array}{@{}c|c@{}}
  a_{11}^{-1} & a_{12}^{-1} \\
\hline
  a_{21}^{-1} & a_{22}^{-1}
\end{array}\right)$
\end{mainbox}
 \section{LU-Decomposition}
\begin{howtobox}{LU-Decomposition}
The \textcolor{red}{LU-Decomposition} is a tool to\\
solve SLE. It does this by factorizing a matrix, making\\
it easy to solve the same matrix vor different RHS.\\
\begin{tabular}{l|l}
	1. Find $PA = LR$ & $\left[\arraycolsep=3.8pt\begin{array}{ccc|ccc|ccc}
	1 & 0 & 0 & 2 & 1 & 2 & 1 & 0 & 0\\
	0 & 1 & 0 & 1 & 2 & 3 & 0 & 1 & 0\\
	0 & 0 & 1 & 2 & 2 & 2 & 0 & 0 & 1\\
	\end{array}\right] $\\
	\makecell[l]{2. Solve $Lc = Pb$\\\quad(forward subst.)} & $\left[\arraycolsep=3.64pt\begin{array}{ccc|ccc|ccc}
	1 & 0 & 0 & 2 & 1 & 2 & 1 & 0 & 0\\
	0 & 1 & 0 & 0 & \frac{3}{2} & 2 & \textcolor{red}{\frac{1}{2}} & 1 & 0\\
	0 & 0 & 1 & 0 & 1 & 0 & \textcolor{red}{1} & 0 & 1\\
	\end{array}\right] $\\
	\makecell[l]{3. Solve $Rx = c$\\\quad(backward subst.)} & $\left[\arraycolsep=3pt\begin{array}{ccc|ccc|ccc}
	1 & 0 & 0 & 2 & 1 & 2 & 1 & 0 & 0\\
	0 & 1 & 0 & 0 & \frac{3}{2} & 2 & \textcolor{red}{\frac{1}{2}} & 1 & 0\\
	\undermat{P}{0 & 0 & 1} & \undermat{R}{0 & 0 & -\frac{4}{3}} & \undermat{L}{\textcolor{red}{1} & \textcolor{red}{\frac{2}{3}} & 1}\\
	\end{array}\right] $\\
	& \\
	\makecell[l]{If rows are swapped\\ P gets permutated.} & \\
	& \\
	\multicolumn{2}{l}{\makecell[l]{\textcolor{red}{partial pivoting} as a \textcolor{red}{pivot}\\\textcolor{red}{strategy} to minimize rounding errors}}\\
\end{tabular}
\end{howtobox}
 \section{Vector Spaces}
\begin{mainbox}{General}
\setlength{\tabcolsep}{2pt}
\begin{tabular}{rlll}
	Def. & \multicolumn{3}{l}{A \textcolor{blue}{vector space V} over $\mathbb{E}$ is a non-empty set,}\\
	& \multicolumn{3}{l}{on which addition and scalar multiplication}\\
	& \multicolumn{3}{l}{are defined.}\\
	\rule{0pt}{3ex} 
	\textcolor{blue}{Axioms} & $(V1)$ & $x+y=y+x$ & $(\forall x,y \in V)$\\
	& $(V2)$ & $(x+y)+z=x+(y+z)$ & $(\forall x,y,z \in V)$\\
	& $(V3)$ & \multicolumn{2}{l}{$\exists o\in V: x + o = x$}\\
	& & \multicolumn{2}{l}{$(\forall x \in V)$\quad\textcolor{red}{zero vector}}\\
	& $(V4)$ & $\forall x\;\exists (-x): x+(-x) = o$ & $(\forall x \in V)$\\
	& $(V5)$ & \multicolumn{2}{l}{$\alpha(x+y)=\alpha x+\alpha y$}\\
	& & \multicolumn{2}{l}{$(\forall \alpha \in \mathbb{E},\; \forall x,y \in V)$}\\
	& $(V6)$ & \multicolumn{2}{l}{$(\alpha+\beta)x=\alpha x+\beta x$}\\
	& & \multicolumn{2}{l}{$(\forall \alpha, \beta \in \mathbb{E},\;\forall x \in V)$}\\
	& $(V7)$ & \multicolumn{2}{l}{$(\alpha \beta)x = \alpha (\beta x)$}\\
	& & \multicolumn{2}{l}{$(\forall \alpha, \beta \in \mathbb{E},\;\forall x \in V)$}\\
	& $(V8)$ & $1x = x$ & $(\forall x \in V)$\\
	\rule{0pt}{3ex}
	\textcolor{blue}{T 4.1} & \multicolumn{3}{l}{$\forall \alpha \in \mathbb{E}, \forall x,y \in V$}\\
	\multicolumn{1}{r}{i)} & \multicolumn{1}{l}{$0\cdot x = o$} & \multicolumn{1}{r}{ii)} & \multicolumn{1}{l}{$\alpha o = o$}\\
	\multicolumn{1}{r}{iii)} & \multicolumn{3}{l}{$\alpha \cdot x = o\;\Rightarrow\;x=o\;or\;\alpha = 0$}\\
	\multicolumn{1}{r}{iv)} & \multicolumn{3}{l}{$(-\alpha)\cdot x = \alpha \cdot (-x) = -(\alpha x)$}\\
	\rule{0pt}{3ex}
	\textcolor{blue}{T 4.2} & \multicolumn{3}{l}{$\forall x,y \in V,\;\exists z \in V$}\\
	& \multicolumn{3}{l}{$x + z = y$, where z is definite and $z = y + (-x)$}\\
\end{tabular}
\end{mainbox}

\begin{mainbox}{Fields}
\setlength{\tabcolsep}{2pt}
\begin{tabular}{rlll}
	Def. & \multicolumn{3}{l}{A \textcolor{blue}{Field} is a non-empty set $\mathbb{K}$,}\\
	& \multicolumn{3}{l}{on which addition and multiplication}\\
	& \multicolumn{3}{l}{are defined.}\\
	\rule{0pt}{3ex} 
	\textcolor{blue}{Axioms} & $(K1)$ & $\alpha +\beta =\beta +\alpha$ & $(\forall \alpha, \beta \in \mathbb{K})$\\
	& $(K2)$ & $(\alpha +\beta )+\gamma =\alpha +(\beta +\gamma )$ & $(\forall \alpha,\beta,\gamma \in \mathbb{K})$\\
	& $(K3)$ & \multicolumn{2}{l}{$\exists o\in V: \alpha + o = \alpha$}\\
	& & \multicolumn{2}{l}{$(\forall \alpha \in \mathbb{K})$\quad\textcolor{red}{zero element}}\\
	& $(K4)$ & $\forall \alpha\;\exists (-\alpha): \alpha+(-\alpha) = o$ & $(\forall \alpha \in \mathbb{K})$\\
	& $(K5)$ & $\alpha\cdot\beta=\beta\cdot\alpha$ & $(\forall \alpha, \beta \in \mathbb{K})$\\
	& $(K6)$ & \multicolumn{2}{l}{$(\alpha\cdot\beta)\cdot\gamma = \alpha\cdot(\beta\cdot\gamma)$}\\
	& & \multicolumn{2}{l}{$(\forall \alpha, \beta, \gamma \in \mathbb{K})$}\\
	& $(K7)$ & \multicolumn{2}{l}{$\exists\;1 \in \mathbb{K}: \alpha \cdot 1 = \alpha$}\\
	& & \multicolumn{2}{l}{$(\forall \alpha \in \mathbb{K})$\quad\textcolor{red}{identity element}}\\
	& $(K8)$ & \multicolumn{2}{l}{$\forall \alpha \in \mathbb{K},\;\alpha \neq 0,\;\exists \alpha^{-1} \in \mathbb{K}:$}\\
	& & \multicolumn{2}{l}{$\alpha \cdot (\alpha^{-1}) = 1$\quad\textcolor{red}{inverse}}\\
	& $(K9)$ & \multicolumn{2}{l}{$\alpha\cdot(\beta + \gamma) = \alpha\cdot\beta + \alpha\cdot\gamma$}\\
	& & \multicolumn{2}{l}{$(\forall \alpha, \beta, \gamma \in \mathbb{K})$}\\
	& $(K10)$ & \multicolumn{2}{l}{$(\alpha + \beta)\cdot\gamma) = \alpha\cdot\gamma + \beta\cdot\gamma$}\\
	& & \multicolumn{2}{l}{$(\forall \alpha, \beta, \gamma \in \mathbb{K})$}\\
\end{tabular}
\end{mainbox}

\begin{mainbox}{Subspaces}
\setlength{\tabcolsep}{2pt}
\begin{tabular}{rl}
	Def. & A \textcolor{blue}{subspace U} is a non-empty subset of\\
	& a vector space V, which is closed under sums\\
	& and scalar multiples.\\
	\rule{0pt}{3ex} 
	\textcolor{blue}{T 4.3} & A subspace is a vector space itself.\\
	\rule{0pt}{3ex} 
	\textcolor{blue}{T 4.4} & With $A\in\mathbb{R}^{mxn}$ and $L_0$ containing \\
	& solutions of $Ax = o$, $L_0$ is a subspace of $\mathbb{R}^n$.\\
	\rule{0pt}{3ex} 
	Def. & The set of all linear combinations of $v_1, ..., v_n$\\
	& is a subspace spanned by these vectors.\\
	& \textcolor{red}{$span\{v_1, ..., v_n\}$} / linear hull of $v_1, ..., v_n$\\
	\rule{0pt}{3ex} 
	Def. & The vectors $v_1, ..., v_n$ are a \textcolor{blue}{spanning set}\\
	& of V, if $\forall w \in V\:\Rightarrow\;w\in span\{v_1, ..., v_n\}$.
\end{tabular}
\end{mainbox}

\begin{mainbox}{Linear Dependencies, Bases, Dimensions}
\setlength{\tabcolsep}{2pt}
\begin{tabular}{rl}
	Def. & Vectors $v_1, ..., v_n$ are \textcolor{blue}{linearly independent}, if no\\
	& vector is a linear combination of the others.\\
	& $\Sigma_{k=0}^{n}\;\alpha_k v_k = 0$, only if $\alpha_1 = ... = \alpha_k = 0$ \\
	\rule{0pt}{3ex} 
	Def. & A $span\{v_1, ..., v_n\} = V$ is a \textcolor{blue}{basis} of V,\\
	& if $v_1, ..., v_n$ are linearly independent.\\ 
	& $\Rightarrow$ standard basis consists of unit vectors\\
	\rule{0pt}{3ex} 
	Def. & The \textcolor{blue}{dimension} of V is denoted as,\\
	& \textcolor{blue}{$dimV$} $=|spanV|$.\quad$dim\{0\} = 0$\\
	\rule{0pt}{3ex} 
	\textcolor{blue}{L 4.8} & Any set $\{v_1, ..., v_m\}\;\subset V$ with $|B_V| < m$ \\
	& is linearly dependent.\\
	\rule{0pt}{3ex} 
	\textcolor{blue}{T 4.9} & Any set of linearly independent vectors of V\\
	& can be extended to a basis of V.\\
	& (as long as V has a finite spanning set)\\
	\rule{0pt}{3ex} 
	\textcolor{blue}{C 4.10} & The set of n linearly independent vectors is a\\
	& basis of V in any finite vector space, if $dimV = n$.\\
	\rule{0pt}{3ex}
	Def. & The coefficients $\xi_k$ are \textcolor{blue}{coordinates} of x\\
	& in basis B. $\xi = (\xi_1 ... \xi_n)^T$ is the \textcolor{blue}{coordinate vector}\\
	& and $x = \Sigma_{i=1}^n\;\xi_ib_i$ is the representation of x\\
	& in coordinates of B.\\
	\rule{0pt}{3ex}
	Def. & Two subspaces $U,\;U'\;\subset\;V$ are complementary,\\
	& if every $v\in V$ has a specific representation \\
	& $v=u+u'$, with $u\in U,\; u' \in U'$. In that case\\
	& V is the \textcolor{blue}{direct sum} of $U$ and $U'$:\\
	& $V = U \bigoplus U'$.\\
\end{tabular}
\end{mainbox}

\begin{mainbox}{Change Of Basis, Coordinate Transformation}
\setlength{\tabcolsep}{2pt}
\begin{tabular}{rl}
	Def. & To change from an old basis $B$ to a new basis $B'$\\
	& one can get the new basis vectors $b_k'$ as follows:\\
	& $b_k' = \Sigma_{i=1}^n\;\tau_{ik}b_i$\\
	& $T = (\tau_{ik})$ is called \textcolor{blue}{transformation matrix}.\\
	\rule{0pt}{3ex}
	\textcolor{blue}{T 4.13} & $\xi = T\xi'$ and $\xi' = T^{-1}\xi$. T is invertible (non-singular)\\
	\rule{0pt}{3ex}
	Def. & $B' = B\cdot T$ to get the new basis.\\
\end{tabular}
\end{mainbox}
 \section{Linear Maps}
\begin{mainbox}{General}
	\setlength{\tabcolsep}{2pt}
	\begin{tabular}{rl}
		Def. & A transformation $F: X \rightarrow Y,\; x\mapsto F(x)$ is \textcolor{blue}{linear},\\
		& if the following holds:\\
		& $F(x+\tilde{x}) = F(x) + F(\tilde{x})$\\
		& $F(\gamma x) = \gamma F(x)$\\
	\end{tabular}
\end{mainbox}

\begin{mainbox}{Functions}
	\setlength{\tabcolsep}{2pt}
	\begin{tabular}{rll}
		Def. & \textcolor{blue}{Injective}: & $\forall x, x' \in X: F(x) = F(x') \Rightarrow x = x'$ \\
		& \textcolor{blue}{Surjective}: & $F(X) = Y$ \\
		& \textcolor{blue}{Bijective}: & injective and surjective $\Rightarrow\;f^{-1}$ existiert \\
	\end{tabular}
\end{mainbox}

\begin{mainbox}{Transformation matrix}
	\setlength{\tabcolsep}{2pt}
	\begin{tabular}{rl}
		\multicolumn{2}{l}{Let $F$ be a linear transformation $X \rightarrow Y$. $F(b_l) \in Y$ can}\\
		\multicolumn{2}{l}{be written using a linear combination of the basis of Y:}\\
		\multicolumn{2}{l}{$F(b_l) = \Sigma_{k=1}^m a_{kl}c_k$}\\
		\rule{0pt}{3ex}
		Def. & The mxn matrix $A = (a_{kl})$ is the \textcolor{blue}{matrix for F}\\
		& \textcolor{blue}{relative to the given bases} in X and Y.\\
		& \\
		& \multicolumn{1}{l}{
		\begin{tabular}{ll}		
		\begin{tabular}{l}
			$F(x) = y \Leftrightarrow A\xi = \eta$\\
		\end{tabular} & \begin{tabular}{l}
			\tikzstyle{line} = [draw, -latex']
\begin{tikzpicture}[
	node distance=1cm and 2cm,
  	>=latex',
  	auto,
	thick
  ]
	\node (X) {$x\in X$};
  	\node[right = of X] (Y) {$y\in Y$};
  	\node[below = of Y] (Em) {$\eta\in\mathbb{E}^m$};
  	\node[below = of X] (En) {$\xi\in\mathbb{E}^n$};

	\path[->]
	(X) edge node {F} (Y)
	(En) edge node {A} (Em);
	\path[->, shift left=.75ex]
    (Y) edge node {$k_y$} (Em)
    (Em) edge node {$k_y^{-1}$} (Y)
    (X) edge node {$k_x$} (En)
    (En) edge node {$k_x^{-1}$} (X);
\end{tikzpicture}\\
		\end{tabular}
		\end{tabular}}\\
		\rule{0pt}{3ex}
		Def. & If $F$ is bijective, we call it an \textcolor{blue}{isomorphism},\\
		& if $X = Y$, we call it an \textcolor{blue}{automorphism}.\\
		\rule{0pt}{3ex}
		\textcolor{blue}{L 5.1} & If $F$ is an isomorphism, there exists $F^{-1}$\\
		& and $F^{-1}$ is also an isomorphism.\\
	\end{tabular}
\end{mainbox}

\begin{mainbox}{Kernel, Image and Rank}
\setlength{\tabcolsep}{2pt}
\begin{tabular}[t]{rll}
	Def. & \multicolumn{2}{l}{The \textcolor{blue}{kernel} of F $ker\;F :\equiv \{x\in X | F(x) = o\}$}\\
	& \multicolumn{2}{l}{is a subspace of X. \textcolor{red}{F injective $\Leftrightarrow ker\;F = \{o\}$ (T 5.6)}}\\
	\rule{0pt}{3ex}
	Def. & \multicolumn{2}{l}{The \textcolor{blue}{image} of F $im\;F :\equiv \{F(x) | x\in X\}$}\\
	& \multicolumn{2}{l}{is a subspace of Y. \textcolor{red}{F surjective $\Leftrightarrow im\;F = Y$}}\\
	\rule{0pt}{3ex}
	\textcolor{blue}{T 5.7} & \multicolumn{2}{l}{\textcolor{blue}{Rank-nullity theorem}:}\\
	& \multicolumn{2}{l}{$dim\;X - dim(ker\;F) = dim(im\;F) = rankF$}\\
	\rule{0pt}{3ex}
	Def. & \multicolumn{2}{l}{The \textcolor{blue}{rank F} is defined as: $rankF = dim(im\;F)$.}\\
	\rule{0pt}{3ex}
	\textcolor{blue}{K 5.8} & $\bullet F:\; X\rightarrow Y$ injective & $\Leftrightarrow rankF = dim\;X$\\
	 & $\bullet F:\; X \rightarrow Y$ surjective & $\Leftrightarrow rankF = dim\;Y$\\
	 & $\bullet F:\; X \rightarrow Y$ bijective & $\Leftrightarrow rankF = dim\;X = dim\;Y$\\
	 & \;\;\textcolor{red}{Isomorphism} & \\
	 & $\bullet F:\; X \rightarrow X$ bijective & $\Leftrightarrow rankF = dim\;X$\\
	 & \;\;\textcolor{red}{Automorphism} & $\Leftrightarrow ker\;F = o$\\
	 \rule{0pt}{3ex}
	 Def. & \multicolumn{2}{l}{Two vector spaces X, Y are isomorphic, if}\\
	 & \multicolumn{2}{l}{there exists an isomorphism $F:\;X\rightarrow Y$.}\\
	 \rule{0pt}{3ex}
	 \textcolor{blue}{T 5.9} & \multicolumn{2}{l}{Two finite vector spaces X, Y are isomorphic}\\
	 & \multicolumn{2}{l}{$\Leftrightarrow dim\;X = dim\;Y$}\\
	 \rule{0pt}{3ex}
	 \begin{tabular}{l}	\textcolor{blue}{C 5.10} \\
	 		\\
	 		\\
	 	\end{tabular} & \multicolumn{2}{l}{\begin{tabular}{rl}
	 		i) & $rank(G\circ F) \leq min\{rankF, rankG\}$\\
	 		ii) & $G$ injective $\Rightarrow rank(G\circ F) = rankF$\\
	 		iii) & $G$ surjective $\Rightarrow rank(G\circ F) = rankG$\\
	 	\end{tabular}}\\
\end{tabular}
\end{mainbox}

\begin{mainbox}{Matrices as Linear Mappings}
\setlength{\tabcolsep}{2pt}
\begin{tabular}{rl}
	Def. & The \textcolor{blue}{column space / range} of A is the subspace\\
	& $R(A) = span\{a_1, ..., a_n\} = im\;A$\\
	\rule{0pt}{3ex}
	Def. & The \textcolor{blue}{null space} of A is the subspace\\
	& $N(A) = L_0(Ax = o) = ker\;A$\\
	& \#free\_parameters $= dim(N(A))$\\
	\rule{0pt}{3ex}
	\textcolor{blue}{T 5.12} & With $rankA = r$ and solution set $L_0$ of $Ax = o$:\\
	& $dim\;L_0\equiv dim\;N(A)\equiv dim(ker\;A) = n - r$ \\
	\rule{0pt}{3ex}
	\textcolor{blue}{T 5.13} & for $A^{mxn}$: $rankA\;=\;...$ \\
	& \begin{tabular}{rl}
		i) & \#pivotelements in reduced form of A\\
		ii) & $dim(im\;A)$\\
		iii) & $dim(column\;space)$\\
		iv) & $dim(row\;space)$\\
	\end{tabular}\\
	\rule{0pt}{3ex}
	\textcolor{blue}{C 5.14} & $rankA^T = rankA^H = rankA$\\
	\rule{0pt}{3ex}
	\textcolor{blue}{T 5.16} & With $A\in \mathbb{E}^{mxn}$, $B\in \mathbb{E}^{pxm}$:\\
	& \begin{tabular}{rl}
		i) & $rank(BA) \leq min\{rankB, rankA\}$\\
		ii) & $rankB = m (\leq p) \Rightarrow rank(BA) = rankA$\\
		iii) & $rankA = m (\leq n) \Rightarrow rank(BA) = rankB$\\
	\end{tabular}\\
\end{tabular}
\end{mainbox}

\begin{mainbox}{}
\setlength{\tabcolsep}{2pt}
\begin{tabular}{rl}
	\textcolor{blue}{T 5.18} & For square matrices the following statements\\
	& are equivalent:\\
	& \begin{tabular}{rlrl}
		i) & A is invertible & ii) & A is non-singular \\
		\rule{0pt}{2ex}
		iii) & $rankA = n$ & iv) & columns are lin.\\
		\rule{0pt}{2ex}
		v) & rows are linearly & & independent\\
		\rule{0pt}{2ex}		
		& independent & vi) & $im\;A \equiv R(A) = \mathbb{E}^n$\\
		\rule{0pt}{2ex}
		vii) & $ker\;A \equiv N(A) = \{o\}$ & viii) & $A:\mathbb{E}^n\rightarrow\mathbb{E}^n$ is an\\
		\rule{0pt}{2ex}
		ix) & A is the transformation & & automorphism\\
		\rule{0pt}{2ex}
		& matrix for a coordinate & & \\
		\rule{0pt}{2ex}
		& transformation & & \\
		\rule{0pt}{2ex}
	\end{tabular}\\
\end{tabular}
\end{mainbox}

\begin{mainbox}{Affine Spaces and solutions to inhomogeneous LSE}
\setlength{\tabcolsep}{2pt}
\begin{tabular}{rl}
	Def. & Let $U$ be a subspace of $V$ and $u_0 \in V$:\\
	& $u_0 + U :\equiv \{u_0 + u | u\in U\}$ is an \textcolor{blue}{affine (sub)space}.\\
	\rule{0pt}{3ex}
	Def. & Let $F:\;X\rightarrow Y$ be a linear mapping and $y_0 \in Y$:\\
	& $H:\;X\rightarrow y_0+Y$, $x\mapsto y_0 + F(x)$ is an \textcolor{blue}{affine mapping}.\\
	\rule{0pt}{3ex}
	\textcolor{blue}{T 5.19} & Let $x_0$ be any solution of $Ax = b$ and\\
	& let $L_0$ be the solution set for $Ax = o$: \\
	& The solution set $L_b$ to $Ax = b$ is an affine subspace:\\
	& $L_b = x_0 + L_0$\\
\end{tabular}
\end{mainbox}

\begin{mainbox}{The Transformation matrix for coord. transformation}
\setlength{\tabcolsep}{2pt}
\begin{tabular}{ll}
	\multicolumn{2}{l}{Let $\;X$, $\;Y$ be vector spaces with $dim\;X = n$, $dim\;Y = m$:}\\
	F:$\;X\rightarrow Y,\;x\mapsto y$ & a linear map,\\
	A:$\;\mathbb{E}^n\rightarrow \mathbb{E}^m,\;\xi\mapsto \eta$ & a (mapping) matrix,\\
	B:$\;\mathbb{E}^n\rightarrow \mathbb{E}^m,\;\xi'\mapsto \eta'$ & a (mapping) matrix,\\
	T:$\;\mathbb{E}^n\rightarrow \mathbb{E}^n,\;\xi'\mapsto \xi$ & a transformation matrix in $\mathbb{E}^n$,\\
	S:$\;\mathbb{E}^m\rightarrow \mathbb{E}^m,\;\eta'\mapsto \eta$ & a transformation matrix in $\mathbb{E}^m$,\\
	\multicolumn{2}{l}{
	\begin{tabular}{ll}
		\begin{tabular}{l} 
			\tikzstyle{line} = [draw, -latex']
\begin{tikzpicture}[
	node distance=1cm and 2cm,
  	>=latex',
  	auto,
	thick
  ]
	\node (X) {$x\in X$};
  	\node[right = of X] (Y) {$y\in Y$};
  	\node[below = of Y] (Em) {$\eta\in\mathbb{E}^m$};
  	\node[below = of X] (En) {$\xi\in\mathbb{E}^n$};
  	\node[below = of En] (En2) {$\xi'\in\mathbb{E}^n$};
  	\node[below = of Em] (Em2) {$\eta'\in\mathbb{E}^m$};

	\path[->]
	(X) edge node {$F$} (Y)
	(En) edge node {A} (Em)
	(En2) edge node {B} (Em2);
	\path[->, shift left=.75ex]
    (Y) edge node {$k_y$} (Em)
    (Em) edge node {$k_y^{-1}$} (Y)
    (X) edge node {$k_x$} (En)
    (En) edge node {$k_x^{-1}$} (X)
    (En) edge node {$\text{T}^{-1}$} (En2)
    (En2) edge node {T} (En)
    (Em) edge node {$\text{S}^{-1}$} (Em2)
    (Em2) edge node {S} (Em);
\end{tikzpicture}
		\end{tabular} & \begin{tabular}{l}
			$\text{A} = \text{S}\cdot\text{B}\cdot\text{T}^{-1}$\\
			$\text{B} = \text{S}^{-1}\cdot\text{A}\cdot\text{T}$\\
			\\
			If $F:\;X\rightarrow X$:\\
			$\text{A} = \text{T}\cdot\text{B}\cdot\text{T}^{-1}$\\
			$\text{B} = \text{T}^{-1}\cdot\text{A}\cdot\text{T}$\\
		\end{tabular}\\
	\end{tabular}}\\
\end{tabular}
\begin{tabular}{rl}
	Def. & Two nxn matrices A and B are \textcolor{blue}{similar},\\
	& if there exists a non-singular matrix T, such that\\
	& either $\text{B} = \text{T}^{-1}\cdot\text{A}\cdot\text{T}$ or $\text{A} = \text{T}\cdot\text{B}\cdot\text{T}^{-1}$.\\
	& \\
\end{tabular}
\end{mainbox}
\end{document}
